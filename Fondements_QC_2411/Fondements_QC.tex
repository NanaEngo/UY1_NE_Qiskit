\makeatletter\def\input@path{
	{./Graphics/}
}\makeatother%%
%
\documentclass[12pt,a4paper]{report} 
\usepackage{graphicx,amsmath,physics,fancyhdr,booktabs,hyperref}
\usepackage[T1]{fontenc}
\usepackage[utf8]{inputenc}
\usepackage[french]{babel}
\usepackage[svgnames,x11names,rgb,table]{xcolor}
\usepackage[margin=2cm,includefoot,includehead]{geometry}
\usepackage[hybrid,texMathDollars,texMathSingleBackslash,rawAttribute]{markdown}
\usepackage{tikz,tikzscale}
\usetikzlibrary{shapes.geometric, arrows, positioning, shadows, fit}
%
\hypersetup{ 
	colorlinks=true, 	% colorise les liens 
	breaklinks=true,	% permet le retour la ligne dans les liens trop longs 
	urlcolor= Blue3,  	% couleur des hyperliens (inclut dans x11names de xcolor
	linkcolor= Blue3, 	% couleur des liens internes 
	citecolor=Blue3,	% couleur des liens de citation
	%   bookmarksopen=true,	% ouvre les signets PDF au départ 
	pdfauthor={Nana Engo},	% Nom de l'auteur du texte
	pdfsubject={Chimie quantique computationnelle},%
	pdfkeywords={Hamiltonien, Approximation BO, HF, Etat fondamental, Etat excité}, % Mots clés
	pdftitle=Fondements et méthodes en chimie quantique computationnelle V2411 % Titre
}
%
\frenchbsetup{StandardLayout}
%
%
\title{\textcolor{blue}{Fondements et méthodes en chimie quantique computationnelle}}
\author{Nana Engo\\serge.nana-engo@facsciences-uy1.cm\\
Département de Physique\\ Université de Yaoundé I}
\date{
 \vspace{5cm} % Whitespace between the title block and the publisher
\begin{center}
\includegraphics[scale=1]{Neo_logo.jpg}
\end{center}
Novembre 2024
}%
%\titlegraphic{\includegraphics{Neo_logo.jpg}}
\begin{document}
\fancypagestyle{logo}{
\fancyhf{}
\fancyhead[LE,LO]{\includegraphics[scale=.3]{Neo_logo.jpg}}
\fancyhead[RE,RO]{\thepage}
\fancyhead[C]{Nana Engo}
}
\pagestyle{logo}

\arrayrulecolor{LawnGreen}\renewcommand{\arraystretch}{1.2}

\maketitle\begin{markdown}


## Fondements de la chimie quantique


En chimie quantique, l'équation fondamentale est celle de Schrödinger dépendant du temps :

\[
i\hbar \frac{\partial}{\partial t} |\psi(t)\rangle = \hat{H} |\psi(t)\rangle
\]

Cette équation décrit comment la fonction d'onde \(|\psi(t)\rangle\) évolue dans le temps sous l'action de l'Hamiltonien \(\hat{H}\). Cependant, résoudre cette équation dans sa forme générale est complexe, même en négligeant les effets relativistes.

Pour simplifier, on se concentre souvent sur l'équation de Schrödinger indépendante du temps, un problème aux valeurs propres :

\[
\hat{H} |\psi\rangle = E |\psi\rangle
\]

Ici, l’hamiltonien \(\hat{H}\) est exprimé comme suit, en unités atomiques (où la charge de l'électron, la masse de l'électron et \(\hbar\) sont égales à 1) (voir la \autoref{fig:interactionqc}):

\[
\hat{H} = -\sum_{i=1}^N \frac{1}{2} \nabla_i^2 - \sum_{A=1}^M \frac{1}{2M_A} \nabla_A^2 - \sum_{i=1}^N \sum_{A=1}^M \frac{Z_A}{r_{iA}} + \sum_{i=1}^N \sum_{j>i} \frac{1}{r_{ij}} + \sum_{A=1}^M \sum_{B>A} \frac{Z_A Z_B}{R_{AB}}
\]

- \(N\) : Nombre d'électrons
- \(M\) : Nombre de noyaux
- \(M_A\) : Rapport entre la masse du noyau \(A\) et celle de l'électron
- \(Z_A\) : Numéro atomique du noyau \(A\)
- \(r_{iA}\) : Distance entre l’électron \(i\) et le noyau \(A\)
- \(R_{AB}\) : Distance entre les noyaux \(A\) et \(B\)

\begin{figure}[tbph]
\centering
\includegraphics[width=0.7\linewidth]{home/taamangtchu/Documents/UY1_NE_Qiskit/Fondements_QC_2411/Graphics/Interaction_QC}
\caption{Illustration simplififée des différentes interactions (électron-noyau, électron-électron, noyau-noyau) dans un système moléculaire.}
\label{fig:interactionqc}
\end{figure}

Les différents termes de l'Hamiltonien sont,

1. **Énergie cinétique des électrons**
   \[
   -\sum_{i=1}^N \frac{1}{2} \nabla_i^2
   \]

2. **Énergie cinétique des noyaux**
   \[
   -\sum_{A=1}^M \frac{1}{2M_A} \nabla_A^2
   \]

3. **Interaction Coulombienne électron-noyau**
   \[
   -\sum_{i=1}^N \sum_{A=1}^M \frac{Z_A}{r_{iA}}
   \]

4. **Répulsion électron-électron**
   \[
   \sum_{i=1}^N \sum_{j>i} \frac{1}{r_{ij}}
   \]

5. **Répulsion noyau-noyau**
   \[
   \sum_{A=1}^M \sum_{B>A} \frac{Z_A Z_B}{R_{AB}}
   \]

L'équation de Schrödinger dans sa forme complète est inapplicable directement pour des systèmes multi-électrons. Des approximations sont nécessaires pour rendre les calculs réalisables.


## Approximation de Born-Oppenheimer

L'**approximation de Born-Oppenheimer** est une hypothèse fondamentale en chimie quantique, qui simplifie le traitement des systèmes moléculaires complexes. Elle repose sur la grande différence de masse entre les noyaux et les électrons. Les noyaux, étant bien plus massifs que les électrons, se déplacent beaucoup plus lentement. Cette différence de dynamique permet de séparer les mouvements électroniques et nucléaires dans l’équation de Schrödinger (voir la \autoref{fig:bointeractionqc}.

\begin{figure}[tbph]
\centering
\includegraphics[width=0.7\linewidth]{Graphics/BO_Interaction_QC}
\caption{Illustration des interactions dams l'approximation de Born-Oppenheimer.}
\label{fig:bointeractionqc}
\end{figure}

### Hamiltonien réduit

En appliquant cette approximation, le hamiltonien électronique est extrait de l'hamiltonien total :

\[
\hat{H}_{\text{elec}} = -\sum_{i=1}^N \frac{1}{2} \nabla_i^2 - \sum_{i=1}^N \sum_{A=1}^M \frac{Z_A}{r_{iA}} + \sum_{i=1}^N \sum_{j>i} \frac{1}{r_{ij}}
\]

Cet hamiltonien électronique dépend explicitement des positions des électrons, mais les positions des noyaux y apparaissent uniquement comme des paramètres fixes. Ainsi, l'énergie électronique \((E_{\text{elec}})\) est calculée pour une configuration donnée des noyaux.

### Énergie totale du système

Une fois l'énergie électronique obtenue, l'énergie totale est exprimée comme suit :

\[
E_{\text{tot}} = E_{\text{elec}} + \sum_{A=1}^M \sum_{B>A} \frac{Z_A Z_B}{R_{AB}}
\]

- \(E_{\text{elec}}\) : Énergie électronique (résultat du hamiltonien électronique).
- Le second terme représente l'énergie de répulsion entre les noyaux, considérés comme des charges ponctuelles fixes.

### Conséquences de l'approximation

- L'énergie totale est calculée pour une géométrie nucléaire fixe.
- Les mouvements des noyaux (vibration, rotation, translation) peuvent être traités séparément en utilisant un hamiltonien nucléaire effectif :

\[
\hat{H}_{\text{nucl}} = -\sum_{A=1}^M \frac{1}{2M_A} \nabla_A^2 + E_{\text{tot}}(\{R_A\})
\]

Ici, \(E_{\text{tot}}(\{R_A\})\) agit comme un potentiel pour le mouvement des noyaux.

En résumé, l'approximation de Born-Oppenheimer est essentielle pour séparer les degrés de liberté nucléaires et électroniques, rendant ainsi le problème beaucoup plus gérable en chimie quantique computationnelle.


## Types d'excitations

En chimie quantique, l'étude des états excités implique de comprendre les différentes façons dont un électron peut être excité dans une molécule. Les excitations électroniques se produisent lorsque des électrons sont transférés d'un orbital occupé vers un orbital inoccupé, et elles influencent directement les propriétés optiques et spectroscopiques des systèmes étudiés.

\begin{figure}[htpb]
\centering
\includegraphics[width=0.7\linewidth]{Graphics/Excited_states}
\caption{Illustration des différences entre \(\Delta E_{\text{vertical}}\) et \(\Delta E_{\text{adiabatic}}\).}
\label{fig:excitedstates}
\end{figure}

Deux types principaux d'énergies d'excitation sont distingués (voir la \autoref{fig:excitedstates}):

1. **Énergie d'excitation verticale (\( \Delta E_{\text{vertical}} \))**
   - Correspond à l'excitation instantanée sans relaxation de la géométrie moléculaire.
   - Est mesurée directement dans les spectroscopies d'absorption.

2. **Énergie d'excitation adiabatique (\( \Delta E_{\text{adiabatic}} \))**
   - Inclut la relaxation complète de la géométrie après excitation.
   - Est généralement inférieure à l'énergie d'excitation verticale.

Les différents types d'excitations électroniques sont les suivantes:

1. **Excitations de valence**
   - Un électron passe d'un orbital de valence (par exemple, HOMO) à un autre orbital de valence (par exemple, LUMO).
   - Exemple typique : HOMO → LUMO.
   - Application : Fréquentes dans les molécules organiques, elles déterminent les transitions UV-visible.

2. **Excitations de transfert de charge (CT - Charge Transfer)**
   - L'électron est transféré entre des orbitales situées dans des régions spatialement séparées de la molécule.
   - Impliquent une séparation importante de charges, et peuvent être intramoléculaires ou intermoléculaires.
   - Application : Cruciales dans les systèmes photovoltaïques ou les complexes de coordination.

3. **Excitations de cœur**
   - Un électron est excité d'un orbital de cœur (profondement lié) à un orbital de valence ou de conduction.
   - Nécessitent de hautes énergies (souvent dans le domaine des rayons X).
   - Application : Utilisées en spectroscopie XANES ou EXAFS pour sonder les environnements locaux.

4. **Excitations de Rydberg**
   - Un électron est excité vers un orbital situé loin du noyau, ayant une fonction d'onde diffuse.
   - Application : Observées dans des atomes ou des molécules diluées à haute énergie.

On retient que les excitations électroniques jouent un rôle central dans la description des propriétés moléculaires. Comprendre leur nature et leur classification permet d’exploiter leurs caractéristiques dans les domaines comme la spectroscopie, les matériaux fonctionnels et la chimie computationnelle.

Même après avoir appliqué l'approximation de Born-Oppenheimer, le problème électronique ne peut pas être résolu analytiquement pour les systèmes multi-électroniques, même pour l'état fondamental, et nécessite donc des approximations. Il existe une multitude de méthodes reposant sur divers types et niveaux d'approximations. Ces méthodes peuvent être classées en deux grandes catégories : celles basées sur la fonction d'onde et celles basées sur la théorie de la fonctionnelle de densité (DFT). Ces dernières, regroupées sous l'égide de la **DFT**, sont de loin les plus couramment utilisées. Cependant, les fonctionnelles de densité sont, en pratique, approximatives, et les résultats peuvent être sensibles au choix de l'approximation fonctionnelle utilisée. En revanche, la DFT présente l'avantage majeur d'un coût computationnel relativement faible, les coûts des fonctionnelles modernes augmentant selon une loi de puissance de troisième ou quatrième ordre ($\mathcal{O}(N^3$ ou $\mathcal{O}(N^4$) par rapport à la taille du système, en fonction de l'approximation et des détails de l'implémentations.

Les méthodes basées sur la fonction d'onde peuvent offrir une précision plus élevée que la DFT, mais au prix d'un coût computationnel accru. En chimie quantique, la plupart des méthodes basées sur la fonction d'onde sont construites à partir de la méthode de Hartree-Fock.

## Méthode de Hartree-Fock

La méthode de Hartree-Fock est une approche fondamentale qui approxime l’interaction électron-électron dans un système moléculaire. Elle repose sur l'idée de champ moyen (**mean-field**), où chaque électron ressent une répulsion moyenne des autres électrons tout en respectant le principe de symétrie antisymétrique de Pauli.

#### Fonction d'onde - Déterminant de Slater

La fonction d'onde dans cette méthode est représentée par un déterminant de Slater, qui impose l'antisymétrie requise pour les électrons, particules de spin-\(\frac{1}{2}\). Il est défini comme suit :

\[
\psi(x_1, \ldots, x_N) = \frac{1}{\sqrt{N!}}
\begin{vmatrix}
\chi_1(x_1) & \chi_2(x_1) & \cdots & \chi_N(x_1) \\
\chi_1(x_2) & \chi_2(x_2) & \cdots & \chi_N(x_2) \\
\vdots & \vdots & \ddots & \vdots \\
\chi_1(x_N) & \chi_2(x_N) & \cdots & \chi_N(x_N)
\end{vmatrix}
\]
où \(N\) est le nombre d'électrons, et \(\chi_i(x_j)\) représente l'orbitale de spin de l'électron \(j\) dans l'état \(i\).

#### Approche variationnelle

L’énergie totale est obtenue par minimisation variationnelle, où l'énergie est exprimée comme fonction des orbitales. L'opérateur de Fock est introduit pour représenter le champ moyen ressenti par un électron :

\[
F(d) = h + \sum_{b} \big(J_b(d) - K_b(d)\big)
\]
avec,
- \(h\) : Terme à un électron (énergie cinétique et interaction électron-noyau).
- \(J_b(d)\) : Terme de Coulomb (répulsion classique électron-électron).
- \(K_b(d)\) : Terme d'échange (correction purement quantique).

#### Équation de Hartree-Fock

En minimisant l'énergie totale, on obtient l'équation de Hartree-Fock pour les orbitales moléculaires :

\[
F|\psi_a\rangle = \epsilon_a|\psi_a\rangle
\]

En exprimant les orbitales comme une combinaison linéaire de fonctions de base, cette équation devient l'équation de Roothaan-Hall :

\[
\mathbf{F}\mathbf{C} = \mathbf{S}\mathbf{C}\epsilon
\]
où :
- \(\mathbf{F}\) : Matrice de Fock
- \(\mathbf{C}\) : Coefficients des orbitales moléculaires
- \(\mathbf{S}\) : Matrice de recouvrement
- \(\epsilon\) : Énergies orbitalaires

#### Procédure auto-cohérente (SCF)

La méthode Hartree-Fock est résolue itérativement par une procédure auto-cohérente (Self-Consistent Field, SCF). À chaque itération :
1. Une estimation initiale des orbitales est utilisée pour construire \(\mathbf{F}\).
2. Les nouvelles orbitales sont obtenues en diagonalant \(\mathbf{F}\).
3. Le processus est répété jusqu'à convergence de l'énergie.

#### Accélération : Méthode DIIS

Pour améliorer la convergence, la méthode DIIS (Direct Inversion in the Iterative Subspace) est souvent utilisée. Elle minimise les résidus de chaque itération en combinant les solutions précédentes :

\[
e_{m+1} = \sum_{i=1}^m c_i e_i
\]

avec des coefficients \(c_i\) obtenus en minimisant la norme des résidus.

### Conclusion

La méthode de Hartree-Fock est la base de nombreuses approches modernes en chimie quantique, servant de point de départ pour des méthodes plus précises qui incorporent la corrélation électronique.

Si vous avez besoin de diagrammes ou d'explications supplémentaires, faites-le-moi savoir !

\end{markdown}